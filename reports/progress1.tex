% Bryan Dixon
% Brendan Kelly
% Mark Lewis-Prazen
% Andy Sayler
% University of Colorado
% Distributed Systems
% Spring 2012

\documentclass[11pt]{article}

\usepackage[text={6.5in, 9in}, centering]{geometry}
\usepackage{graphicx}
\usepackage{url}
\usepackage{listings}
\usepackage{hyperref}

\bibliographystyle{plain}

\hypersetup{
    colorlinks,
    citecolor=black,
    filecolor=black,
    linkcolor=black,
    urlcolor=black
}

\newenvironment{packed_enum}{
\begin{enumerate}
  \setlength{\itemsep}{1pt}
  \setlength{\parskip}{0pt}
  \setlength{\parsep}{0pt}
}{\end{enumerate}}

\newenvironment{packed_item}{
\begin{itemize}
  \setlength{\itemsep}{1pt}
  \setlength{\parskip}{0pt}
  \setlength{\parsep}{0pt}
}{\end{itemize}}

\begin{document}

\title{
  Distributed File Storage Systems\\
  Project Report 1
}
\author{
  Bryan Dixon \and Brendan Kelly \and Mark Lewis-Prazen \and Andy Sayler\\
  \and University of Colorado\\
  \texttt{first.last@colorado.edu}
}
\date{\today}

\maketitle

\begin{abstract}
The new generation of applications requires the processing of terabytes and petabytes of data. This processing has 
largely been achieved by distributed processing architectures and methodologies, driving the rise of companies such as 
Google, Amazon and Yahoo, which have embraced such technologies over the past decade and made them a central part of 
their business models. The increased importance of large scale data processing and a corresponding explosion in 
alternative storage architectures and service strategies requires computer science researchers and professionals to 
become well versed in the available alternatives and trade-offs in making technological architectural choices these 
areas. While storage technology has evolved significantly over the past five years, a deep understanding of both 
technical and business storage issues has lagged behind as many in these fields still view data storage as a mundane 
activity still the purview of back office technocrats. The current trend in distributed cloud storage and the focus of 
delivering IT infrastructure and applications bundled with data storage has led to an “as a service” model being used 
to promote such products.

The intent of our project is to examine the current state of both core storage and related service offerings on both a 
technical and business level to understand (1) what is technically available, (2) what are current research and 
developmental directions in storage technology, and (3) how core storage architectures are being coupled with associated
services to produce bundled storage product offerings that purport to add value above and beyond core offerings. We 
believe that by performing this analysis and the subsequent implementation of a distributed proof of concept storage 
model, we will gain a better understanding of evolving storage technologies as well as the tradeoffs that need to be 
made when implementing bundled storage services in a production environment.

\end{abstract}

\newpage

\section{Problem Definition}
Text change

\section{Introduction}


\section{Related Work}


\nocite{*}
\bibliography{refs}

\end{document}
