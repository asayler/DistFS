% Bryan Dixon
% Brendan Kelly
% Mark Lewis-Prazen
% Andy Sayler
% University of Colorado
% Distributed Systems
% Spring 2012

\documentclass[11pt]{article}

\usepackage[text={6.5in, 9in}, centering]{geometry}
\usepackage{graphicx}
\usepackage{url}
\usepackage{listings}
\usepackage{hyperref}

\bibliographystyle{plain}

\hypersetup{
    colorlinks,
    citecolor=black,
    filecolor=black,
    linkcolor=black,
    urlcolor=black
}

\newenvironment{packed_enum}{
\begin{enumerate}
  \setlength{\itemsep}{1pt}
  \setlength{\parskip}{0pt}
  \setlength{\parsep}{0pt}
}{\end{enumerate}}

\newenvironment{packed_item}{
\begin{itemize}
  \setlength{\itemsep}{1pt}
  \setlength{\parskip}{0pt}
  \setlength{\parsep}{0pt}
}{\end{itemize}}

\begin{document}

\title{
  Distributed File Storage Systems\\
  Project Proposal
}
\author{
  Bryan Dixon \and Brendan Kelly \and Mark Lewis-Prazen \and Andy Sayler\\
  \and University of Colorado\\
  \texttt{first.last@colorado.edu}
}
\date{\today}

\maketitle

\newpage

\section{Problem Statement}
Modern applications require the
processing of terabytes and petabytes of data. This processing has
largely been achieved by distributed architectures and
methodologies, driving the rise of companies such as Google, Amazon
and Yahoo. The increased importance of large scale data processing and a
corresponding explosion in alternative storage architectures
requires computer science researchers and
professionals to be well versed in the available technologies.
Yet, while the technology has evolved
significantly over the past five years, a deep understanding of both
technical and business storage issues has lagged behind.

The intent of our project is to examine the current state of distributed
storage systems and services on both a
technical and business level to understand
(1) what is technically available,
(2) what are current research and developmental directions,
and (3) what business cases and values different technologies bring to
the table.

We believe that performing this analysis and a subsequent implementation of
a distributed proof-of-concept storage model, we will gain a better
understanding of distributed storage technologies and the nature
of implementing these systems and services in a production environment.

\section{Project Approach}
Our project approach will consist of three phases:
\begin{packed_enum}
\item Review the current research and technical literature
  in order to gain an understanding of the state of the art in
  distributed storage systems.
\item Define an appropriate architecture for a proof-of-concept
  distributed storage system that can be implemented and evaluated.
\item Implement the proof-of-concept design and evaluate its performance.
\end{packed_enum}

We have requested EEF funding for the necessary hardware to build
a proof-of-concept platform of our design.
At the conclusion of the project these hardware
resources can be used to enhance available computing resources for the
graduate computer science program.

\section{Evaluation Methodology}
We will evaluate our implemented design by
comparing its performance against popular storage processing and
storage services metrics available in the literature.
A complete list of criteria for evaluation and actual test parameters
will be finalized at the conclusion of the literature review and then
subsequently updated at each project status update.

\nocite{*}
\bibliography{refs}

\end{document}
